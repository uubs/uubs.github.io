\documentclass{article}

\usepackage[UTF8]{ctex}
\usepackage{listings}

\bibliographystyle{plain}
\title{what is computer virus, computer worms, trojan horse. And the ancestors for each of them}

\begin{document}
\maketitle

\section{computer virus}
the most significant character of computer virus is that \textbf{replicates itself} by modifying other computer programs and inserting its own code. 

the first computer virus is \textbf{Creeper virus}, written by Bob Thomas in 1971. The source code for Creeper is not avaliable, according to \cite{website:http://lists.project-wombat.org/pipermail/project-wombat-project-wombat.org/2011-November/006898.html}.

due to the loss of \textbf{Creeper}, I choose another virus to analysis, called \textbf{Elk Cloner},which is the first personal computer virus.

ToDo: the following analysis \textbf{Elk Cloner}

\section{computer worm}
the most significant character of computer worm is the \textbf{replicates itself} in order to spread to other computers.

\emph{what's the difference between computer virus and computer worm}
the first computer worm is \textbf{Reaper}, which actually is used to delete the \textbf{Creeper}. Of course, we can not loose the change to analysis \textbf{Reaper}. Another computer worm needed to analysis is \textbf{Morris worm}, which create by a graduate student Robert Tappan Morris in 1988.

ToDO : analysis \textbf{Reaper} if have the source code. analysis \textbf{Morris}

\section{Trojan horse}

A Trojan horse(in the following we call it Trojan) is a malware that \textbf{ misleads users of its true intent }

ToDo: analysis EGABTR and Gh0st RAT

\bibliography{math}
https://en.wikipedia.org/wiki/Computer_virus
https://en.wikipedia.org/wiki/Computer_worm
\end{document}

