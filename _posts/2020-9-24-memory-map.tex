\documentclass{article}
\usepackage[UTF8]{ctex}

\begin{document}

\title{内存分布}
\maketitle

内存分布\footnote{此处正确的表述应该是x86系列的内存分布} 是一个不断发展的名词,从实模式(real mode),到保护模式(protect mode),再到IA-32e模式(long mode),内存的分布具有不同的机制。以下将从实模式开始,讲述在各个模式下的内存分布。

实模式是伴随着16位x86CPU出现的,保护模式伴随着32位x86CPU出现的,IA-32e模式是伴随64位x86CPU出现的。但并不是说,实模式只在16位上,而不会在32位或64位上出现。各种不同的模式的本质区别是对内存的寻址方式的不同。\footnote{此处可能想法还有写问题,还需要斟酌斟酌}
\section{实模式(real mode)}

\subsection{为什么叫做实模式(real mode)}
 这是因为实模式中使用的地址就是物理内存的地址,也就是说,我们知道某个程序的地址是0x4300,他的物理地址就是0x4300。

\subsection{实模式(real mode)中的内存分布}
在实模式中,地址是由段地址:偏移地址组成的。例如对于16位寄存器,默认数据段$DS:DX$,代码段$CS:IP$,以及其他的段地址寄存器$SS, ES, FS, GS$和通用寄存器构成的段,这里对这部分不详细论述。 \\
实模式下,指令可以是利用16位寄存器操作,也可以使用32位寄存器操作。 \footnote{此处的说法可能有问题}



\section{保护模式(protect mode)}

\section{IA-32e模式(long mode)}

\emph{reference}
\begin{enumerate}
  \item{https://en.wikipedia.org/wiki/Real_mode}
\end{enumerate}
\end{document}
