\documentclass{article}

\usepackage[UTF8]{ctex}
\usepackage{color}
\usepackage{listings}


\title{play with CVE-2014-6271}

\begin{document}
\maketitle
\textbf{source : }{read from secutiry course} \\
\textbf{machine used : } Linux 3.13.0-35-generic 2014 i686 GNU/Linux \\
\textbf{target : } enviroment variables in bash

input command
\begin{lstlisting}
\$ env x='() { :;}; echo vulnerable' bash -c "echo test"
executed echo;
vulnerable
test
\end{lstlisting}
the result satisfy the condition of vulnrability. now let's explain the detail of what this command do.\\

First, what is command \textit{env} do. We can find info easily using \textit{man env}. After that, we obtain,
\begin{lstlisting}
env [OPTION]... [-] [NAME=VALUE]... [COMMAND [ARG]...]
Set each NAME to VALUE in the environment and run COMMAND.
\end{lstlisting}

which means we have a environment variable x. while executing \textit{bash -c "echo test"}, this command first read all environment\footnote{why read and how to read?} variable. When it comes to read x we set before, command \textit{(){ :; }; echo vulnerable} will execute\footnote{why reading environment means execute environment, what about the other environment value?}. then the stdout will print 'vulnerable'.\footnote{what's : mean in shell?}. 

In the previous illustrate, attack command is \textit{echo vulnerable}, which actually does nothing to attack though. But we can change this command to other real malicious command. 

Here is a imaging example to use this vulnerability. Imaging there is a web server using CGI, the HTTP request , saying HTTP\_USER\_AGENT, is often included as environment variables. We can spoof user agent to be something like \textit{'() { :; }; echo foo'}, in which \textit{echo foo} can be malicious command, saying create a vulnerable shell. 

\end{document}
