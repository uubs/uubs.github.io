\documentclass[a4]{article}
\usepackage[UTF8]{ctex}
\usepackage{xcolor}
\usepackage{listings}

\usepackage{geometry}
\geometry{a4paper, scale=0.8}

\lstset{
    columns=fixed,       
    %numbers=left,                                        % 在左侧显示行号
    frame=none, %tb,                                     % 不显示背景边框
    backgroundcolor=\color[RGB]{244,244,244},            % 设定背景颜色:浅晦涩,近透明
    keywordstyle=\color[RGB]{40,40,255},                 % 设定关键字颜色
    numberstyle=\footnotesize\color{darkgray},           % 设定行号格式
    commentstyle=\it\color[RGB]{0,96,96},                % 设置代码注释的格式
    stringstyle=\rmfamily\slshape\color[RGB]{128,0,0},   % 设置字符串格式
    showstringspaces=false,                              % 不显示字符串中的空格
    language=C++,                                        % 设置语言C, C++, bash, Fortran, Python, TeX, make
    basicstyle=\small, %\scriptsize%\tiny%\footnotesize  % basic fontsize in it
    breaklines=true
}

\begin{document}

\section{前言}
首先,我们的机器上已经有安装了docker。通过 \lstinline[language=C++]{docker pull ubuntu:16.04}命令拉下版本为\textit{16.04}的ubuntu操作系统。关于将docker官方镜像源更改为阿里云镜像源,请参考这篇文章\footnote{我是footnote}。

\textbf{LAMP},表示\lstinline[language=C++]{Linux+Apache+MySql/MariaDB+Perl/PHP/Python},是一种搭建服务器的解决方案。在本文中讨论的是ubuntu:16.04+Apache2+Mysql5.7+PHP7

\section{安装LAMP}
我们通过\lstinline[language=C++]{docker pull}拉取了ubuntu:16.04镜像到本地,查看本地镜像如下,
\begin{lstlisting}[language=C++]
uubs@uubs:~\$ docker images
REPOSITORY TAG      IMAGE ID      CREATED             SIZE
ubuntu   16.04      4b22027ede29    6 weeks ago       127MB
\end{lstlisting}

通过docker镜像创建docker容器,
\begin{lstlisting}[language=C++]
uubs@uubs:~\$ docker run -it ubuntu:16.04 /bin/bash 
root@5d4fe8bb2fea:/#  <- 进入容器内部
\end{lstlisting}

好的,我们现在进入了我们创建的容器中,要做的\textbf{第一件事}是, 注\footnote{如果此处不运行此条命令来更新源,那么无法下载包}
\begin{lstlisting}[language=C++]
root@5d4fe8bb2fea:/#  apt-get update
\end{lstlisting}

\subsection{安装MYSQL}
运行一下命令进行安装,
\begin{lstlisting}[language=C++]
root@5d4fe8bb2fea:/# apt-get install -y mysql-server mysql-client libmysqlclient-dev
\end{lstlisting}

安装net-tools查看MySql的运行状态,
\begin{lstlisting}[language=C++]
root@5d4fe8bb2fea:/# apt-get install -y net-tools
\end{lstlisting}

下面我们来启动MySql服务器,关于mysql服务器的其他操作,见\ref{q:mysql-control}
\begin{lstlisting}[language=C++]
root@5d4fe8bb2fea:/# /etc/init.d/mysql start
\end{lstlisting}

查看MySql服务器是否启动,
\begin{lstlisting}[language=C++]
root@5d4fe8bb2fea:/# netstat -tl | grep mysql
tcp        0      0 localhost:mysql         *:*          LISTEN     
\end{lstlisting}

\subsection{安装apache2}
运行一下命令进行安装,
\begin{lstlisting}[language=C++]
root@5d4fe8bb2fea:/# apt-get install apache2
\end{lstlisting}

下面我们来启动apache服务器,关于apache服务器的其他操作,见\ref{q:apache-control}
\begin{lstlisting}[language=C++]
root@5d4fe8bb2fea:/# /etc/init.d/apache2 start
 * Starting Apache httpd web server apache2                                      * 
\end{lstlisting}

查看apache2服务器是否启动,
\begin{lstlisting}[language=C++]
root@5d4fe8bb2fea:/# ps -e | grep apache
   42 ?        00:00:00 apache2
   45 ?        00:00:03 apache2
   46 ?        00:00:03 apache2
\end{lstlisting}

\subsection{安装PHP7.0}
安装过程如下,
\begin{lstlisting}[language=C++]
1. 安装 software-properties-common
root@5d4fe8bb2fea:/# apt-get install -y software-properties-common 

2. 添加php仓库
root@5d4fe8bb2fea:/# add-apt-repository ppa:ondrej/php; apt-get update

3. 清除之前安装的php版本
root@5d4fe8bb2fea:/# a2enmod phpx.x

4. 安装PHP7.2
root@5d4fe8bb2fea:/# apt-get install -y --allow-unauthenticated php7.2

5. 安装常用扩展
root@5d4fe8bb2fea:/# apt-get install -y --allow-unauthenticated php7.2-fpm php7.2-mysql php7.2-curl php7.2-json php7.2-mbstring php7.2-xml  php7.2-intl php7.2-odbc php7.2-cgi

\end{lstlisting}

到第4步输出,
\begin{lstlisting}
NOTICE: Not enabling PHP 7.2 FPM by default.
NOTICE: To enable PHP 7.2 FPM in Apache2 do:
NOTICE: a2enmod proxy\_fcgi setenvif
NOTICE: a2enconf php7.2-fpm
NOTICE: You are seeing this message because you have apache2 package installed.
\end{lstlisting}

到此,我们把该安装的东西都安装好了。

\section{制作容器镜像}

现在,可以通过\lstinline{exit}退出docker。

查看我们刚才关闭的容器,
\begin{lstlisting}
uubs@uubs:~\$ docker ps -l
CONTAINER ID  IMAGE        COMMAND      CREATED       STATUS      PORTS     NAMES
5d4fe8bb2fea  ubuntu:16.04 "/bin/bash"  27 hours ago  Exited (0) 19 seconds ago affectionate_goldstine
\end{lstlisting}

将该容器保存为镜像,
\begin{lstlisting}
uubs@uubs:~\$ docker commit 5d4fe8bb2fea lamp:1.0
\end{lstlisting}

查看镜像,
\begin{lstlisting}
uubs@uubs:~\$ docker images
REPOSITORY             TAG                 IMAGE ID            CREATED             SIZE
lamp                   1.0                 8aecf087ea32        38 seconds ago      733MB
\end{lstlisting}

镜像制作成功。

\section{使用lamp:1.0镜像}

创建lamp:1.0镜像的容器,
\begin{lstlisting}
uubs@uubs:~\$ docker run -it lamp:1.0 /bin/bash
\end{lstlisting}

查看容器的地址,
\begin{lstlisting}
root@1eccf20d5c77:/# ifconfig
inet addr:172.17.0.2  Bcast:172.17.255.255  Mask:255.255.0.0 <- 判断出地址
\end{lstlisting}

在宿主机上利用nmap工具扫描172.17.0.2地址的端口开放情况,此处有错误\footnote{如果按照这种方法来扫描下面开启了端口的容器,依然得到端口关闭的结果,待熟悉nmap后再改正}
\begin{lstlisting}
uubs@uubs:~\$ nmap 172.17.0.2
All 1000 scanned ports on 172.17.0.2 are closed
\end{lstlisting}

得知该地址所有的端口都关闭了。\\
下面考虑在容器中打开端口。

借助IPtables\footnote{通过apt-get install iptables安装}工具打开容器的端口,注意这里使用IPtables需要在创建容器时使用\textit{--privileged}参数,否则将会出现下面的错误,
\begin{lstlisting}
iptables v1.6.0: can't initialize iptables table `filter': Permission denied (you must be root)
\end{lstlisting}

下面使用正确的方式创建容器并进入,
\begin{lstlisting}
uubs@uubs:~\$ docker run -it --priviliged lamp:1.0 /bin/bash
\end{lstlisting}

打开容器端口,此处有个问题\footnote{对于下一步的启动apache2,本步究竟有没有起作用?需要熟悉IPtables工具的使用}
\begin{lstlisting}
root@8d6a581df360:/# /sbin/iptables -I INPUT -p tcp --dport 80 -j ACCEPT
\end{lstlisting}

保存规则,
\begin{lstlisting}
root@8d6a581df360:/# iptables-save
\end{lstlisting}

现在我们便开启了80端口。按照之前提到的方法打开apache服务,通过以下语句验证apache服务是否正常工作,
\begin{lstlisting}
telnet 172.17.0.2 80
Trying 172.17.0.2...
Connected to 172.17.0.2.
Escape character is '^]'.
\end{lstlisting}

显示连接上了apache服务,这是如果我们在浏览其中输入\lstinline{http://172.17.0.2:80/}将会得到apache2的字样。

至此,LAMP镜像搭建完毕。

\section{问题及方案}
\subsection{MYSQL 服务器的操作} 

\label{q:mysql-control}
\emph{启动服务器}
\begin{itemize}
    \item 使用service启动:\lstinline{service mysql start}
    \item 使用mysqld脚本启动:\lstinline{/etc/init.d/mysql start}
    \item 其他
\end{itemize}
\emph{停止服务器}
\begin{itemize}
    \item 使用service启动:\lstinline{service mysql stop}
    \item 使用mysqld脚本启动:\lstinline{/etc/init.d/mysql stop}
    \item 其他
\end{itemize}
\emph{重启服务器}
\begin{itemize}
    \item 使用service启动:\lstinline{service mysql restart}
    \item 使用mysqld脚本启动:\lstinline{/etc/init.d/mysql restart}
    \item 其他
\end{itemize}
\emph{查看服务器是否正在运行}
\begin{lstlisting}[language=C++]
root@5d4fe8bb2fea:/# netstat -tl | grep mysql
tcp        0      0 localhost:mysql         *:*          LISTEN     
\end{lstlisting}
\emph{登录数据库}
\begin{lstlisting}[language=C++]
root@5d4fe8bb2fea:/# mysql -u root -p 
<- 输入之前设置的root用户密码,如果没有设置,则默认为空
\end{lstlisting}

\subsection{apache2 服务器的操作} 

\label{q:apache-control}
\emph{启动服务器}
\begin{itemize}
    \item \lstinline{/etc/init.d/apache2 start}
    \item 其他
\end{itemize}
\emph{停止服务器}
\begin{itemize}
    \item \lstinline{/etc/init.d/apache2 stop}
    \item 其他
\end{itemize}
\emph{重启服务器}
\begin{itemize}
    \item \lstinline{/etc/init.d/apache2 restart}
    \item 其他
\end{itemize}
\emph{查看服务器是否正在运行}
\begin{lstlisting}[language=C++]
root@5d4fe8bb2fea:/# ps -e | grep apache
   42 ?        00:00:00 apache2
   45 ?        00:00:03 apache2
   46 ?        00:00:03 apache2
\end{lstlisting}

\section{下一步}
\begin{itemize}
    \item 使用源码编译安装mysql, apache。
    \item 写出搭建过程的dockerfile
\end{itemize}

\end{document}
