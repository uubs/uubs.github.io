\documentclass{article}
\usepackage[UTF8]{ctex}
\usepackage{xcolor}
\usepackage{listings}

\usepackage{geometry}
\geometry{a4paper, scale=0.8}

\lstset{
    columns=fixed,
    %numbers=left,                                        % 在左侧显示行号
    frame=none, %tb,                                     % 不显示背景边框
    backgroundcolor=\color[RGB]{244,244,244},            % 设定背景颜色:浅晦涩,近透明
    keywordstyle=\color[RGB]{40,40,255},                 % 设定关键字颜色
    numberstyle=\footnotesize\color{darkgray},           % 设定行号格式
    commentstyle=\it\color[RGB]{0,96,96},                % 设置代码注释的格式
    stringstyle=\rmfamily\slshape\color[RGB]{128,0,0},   % 设置字符串格式
    showstringspaces=false,                              % 不显示字符串中的空格
    language=C++,                                        % 设置语言C, C++, bash, Fortran, Python, TeX, make
    basicstyle=\small, %\scriptsize%\tiny%\footnotesize  % basic fontsize in it
    breaklines=true
}

\begin{document}

\section{需要解决的问题}
服务器更换ip地址?
\section{解决方案}
需要更改的配置文件
\begin{lstlisting}
root@Cloud:~# sudo vim /etc/network/interfaces 
填入内容格式:
auto eth0                  #设置自动启动eth0接口
iface eth0 inet static     #配置静态IP
address 192.168.11.88      #IP地址
netmask 255.255.255.0      #子网掩码
gateway 192.168.11.1        #默认网关
\end{lstlisting}

根据填入的内容,查看网卡,掩码,
\begin{lstlisting}
root@Cloud:~# ifconfig
ens3: flags=4163<UP,BROADCAST,RUNNING,MULTICAST>  mtu 1500
        inet 45.141.xxx.xx  netmask 255.255.255.0  broadcast 45.141.xxx.255
\end{lstlisting}

查看网关,
\begin{lstlisting}
root@Cloud:~# ip r 
default via 45.141.137.1 dev ens3 proto static
\end{lstlisting}

更改保存,安装network-manager\footnote{sudo apt-get install network-manager}

重启网络,
\begin{lstlisting}
root@Cloud:~# service network-manager restart
\end{lstlisting}

依然出错。
\end{document}
